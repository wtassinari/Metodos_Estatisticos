\documentclass[a4paper,11pt,twoside,openright]{report}
% \documentclass[15pt]{report}

\usepackage[brazil]{babel}    % d� suporte para os termos na l�ngua portuguesa do Brasil
\usepackage[latin1]{inputenc} % d� suporte para caracteres especiais como acentos e cedilha
\usepackage[T1]{fontenc}      % L� a codifica��o de fonte T1 (font encoding default � 0T1).
\usepackage[dvipdfm]{graphicx} % para inclus�o de figuras (png, jpg, gif, bmp)
\usepackage{graphics}          % figuras gr�ficas
\usepackage{color}             % para letras e caixas coloridas
\usepackage{makeidx}           % �ndice remissivo
\usepackage{a4wide}            % correta formata��o da p�gina em A4
\usepackage{setspace}          % para a dist�ncia entre linhas


\begin{document}
\begin{center}
\titulo{\Large {Lista de Exerc\'{\i}cios 02 \\ M\'{e}todos Estat\'{\i}sticos Aplicados as Ci\^{e}ncias Veterin\'{a}rias}
             \\ Professor Wagner Tassinari 
             \\ \small{E-mail: \textit{wtassinari@gmail.com}}\\
\begin{flushleft}
\textbf{Aluno: .............................................................................................}
\end{flushleft}
\end{center} 

% \pagebreak

% \begin{verbatim}
%  http://groups.google.com/group/estatistica-ciencias-veterinarias?hl=pt-BR
% \end{verbatim}

\section*{Quest\~{a}o 1} 
\hspace{0.5cm}Diferencie Popula\c{c}\~{a}o de Amostra. Cite exemplos pr\'{a}ticos de ambas. 

\section*{Quest\~{a}o 2} 
\hspace{0.5cm}Diferencie Amostragem Probabil\'{\i}stica da N\~{a}o-Probabil\'{\i}stica. Cite exemplos de aplica\c{c}\~{a}o. 

\section*{Quest\~{a}o 3} 
\hspace{0.5cm}Diferencie popula\c{c}\~{o}es finitas e infinitas. 

\section*{Quest\~{a}o 4} 
\hspace{0.5cm}Por que a Distribui\c{c}\~{a}o de Probabilidades Normal \'{e} a mais importante na \'{a}rea de estat\'{\i}stica ? 


\section*{Exerc\'{\i}cio 5}
\hspace{0.5cm} Uma pesquisa \'{e} planejada para determinar as despesas m\'{e}dicas anuais das fam\'{\i}lias dos empregados
de uma grande empresa. A ger\^{e}ncia da empresa deseja ter $95\%$ de confian\c{c}a de que a m\'{e}dia da amostra
 est\'{a} no m\'{a}ximo com uma margem de erro de $\pm R\$50$ da m\'{e}dia real das despesas
m\'{e}dicas familiares. Um estudo-piloto indica que o desvio-padr\~{a}o pode ser
calculado como sendo igual a $\pm R\$400$.

\begin{enumerate}
 \item Qual o tamanho de amostra necess\'{a}rio $?$
\item Se a ger\^{e}ncia deseja estar certa em uma margem de erro de $\pm R\$25$, que tamanho de amostra ser\'{a} necess\'{a}rio?
\end{enumerate}


\section*{Exerc\'{\i}cio 6}
\hspace{0.5cm} O teste de QI padr\~{a}o \'{e} planejado de modo que a m\'{e}dia seja 100 e o
desvio-padr\~{a}o para adultos normais seja 15. Ache o tamanho da amostra necess\'{a}ria para estimar 
o QI m\'{e}dio dos instrutores de estat\'{\i}stica. Queremos ter $95\%$ de confian\c{c}a em que nossa 
m\'{e}dia amostral esteja a menos de 1,5 pontos de QI da verdadeira m\'{e}dia. A m\'{e}dia para esta
popula\c{c}\~{a}o \'{e} obviamente superior a 100, e o desvio-padr\~{a}o \'{e} provavelmente
inferior a 15, porque se trata de um grupo com menor varia\c{c}\~{a}o do que um
grupo selecionado aleatoriamente da popula\c{c}\~{a}o geral; portanto, se
tomamos $\sigma = 15$, estaremos sendo conservadores, por utilizarmos um valor que dar\'{a} um tamanho de amostra no
 m\'{\i}nimo t\~{a}o grande quanto necess\'{a}rio. Suponha $\sigma = 15$ e determine o tamanho da amostra necess\'{a}rio.


\section*{Exerc\'{\i}cio 7}
\hspace{0.5cm}Um assistente social deseja saber o tamanho da amostra ($n$) necess\'{a}rio para determinar a
propor\c{c}o da popula\c{c}\~{a}o atendida por uma Unidade de Sa\'{u}de, que pertence ao munic\'{\i}pio de
Cariacica. N\~{a}o foi feito um levantamento pr\'{e}vio da propor\c{c}\~{a}o amostral e, portanto, seu valor \'{e}
desconhecido. Ela quer ter $95\%$ de confian\c{c}a que sua o erro m\'{a}ximo da estimativa ($\varepsilon$) seja
de $\pm 5\%$ (ou $0,05$). Quantas pessoas necessitam ser entrevistadas?

\section*{Exerc\'{\i}cio 8}
\hspace{0.5cm}No banco de dados \textbf{bernes.csv}, verifique se as vari\'{a}veis: umidade, precipita\c{c}\~{a}o e n\'{u}mero de bernes seguem uma distribui\c{c}\~{a}o normal.


\section*{Exerc\'{\i}cio 9}
\hspace{0.5cm}No banco de dados \textbf{nocardia.csv}, verifique se as vari\'{a}veis: m\'{e}dia da produ\c{c}\~{a}o de leite do rebanho (kg/cow/day) ("prod")	e a utiliza\c{c}\~{a}o de neomycin ("dneo") na fazenda no \'{u}ltimo ano seguem uma distribui\c{c}\~{a}o normal.

\vspace{20mm} 
\small{\textbf{OBS:} Nas \'{u}ltimas duas quest\~{o}es, utilize os m\'{e}todos visuais gr\'{a}ficos via histograma. Explore tamb\'{e}m os testes de Shapiro-Wilk e outros de normalidade. 
Caso a vari\'{a}vel n\~{a}o se apresente normalmente distribuida, utilize algumas t\'{e}cnicas de transforma\c{c}\~{a}o de vari\'{a}veis. Comente os resultados.}

\end{document}